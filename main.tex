% ---------------- RELAZIONE PROGETTO DI PROGRAMMAZIONE AD OGGETTI (OOP) --------
\documentclass[a4paper,12pt]{report}

% ----------------------------- PREAMBLE --------------------------------------- 

\usepackage{lmodern}
\usepackage{alltt, fancyvrb, url}
\usepackage{float}
\usepackage{graphicx}
\usepackage[utf8]{inputenc}
\usepackage{hyperref}
\usepackage{amsmath,amssymb,amsthm}

% Questo commentalo se vuoi scrivere in inglese.
\usepackage[italian]{babel}

\usepackage[italian]{cleveref}

\usepackage{comment}
\usepackage{microtype}
\usepackage{fancyhdr}

\usepackage[scaled=.92]{helvet}
\usepackage[T1]{fontenc}

\usepackage{lscape}

% hyperref settings
\hypersetup{
	colorlinks=true,
	linkcolor=black, %blue
	filecolor=magenta,      
	urlcolor=cyan,
	pdftitle={Sharelatex Example},
	bookmarks=true,
	pdfpagemode=FullScreen,
}

% ----------------------------- PREAMBLE END -----------------------------------

\makeindex

\title{\textbf{INGEGNERIA DEL SOFTWARE}}
\author{Alessandro Pioggia}

\begin{document}
	
	\makeatletter
	\begin{titlepage}
		\begin{center}
			{\Huge  \@title }\\[3ex] 
			{\large  \@author}\\[3ex] 
			{\large \@date}
		\end{center}
	\end{titlepage}
	\makeatother
	\thispagestyle{empty}
	\newpage
	
	%\maketitle
	
	\tableofcontents
	
	% \input: import the commands from filename.tex to target file.
	
	% \include: does a \clearpage and does an \input.
	
	\newpage
	
	\section{Analisi dei requisiti}
	L'analisi dei requisiti è la fase che permette, attraverso la modellazione della realtà, di redigere le specifiche dei requisiti. Per rendere possibile la creazione un coadiuvato è necessario, da parte del produttore, intervistare il consumatore, cercando di assimilare al meglio tutte le informazioni riguardanti il dominio applicativo. \\
	In questa fase è necessario fare in modo che il documento sia chiaro, non ambiguo e accessibile a tutti (deve quindi essere comprensibile anche a "non tecnici"). 
	
	
	
	
	
\end{document}
